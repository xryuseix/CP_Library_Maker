\documentclass[a4j,titlepage]{jarticle} %ここは関係ない
\usepackage{listings,jlisting} %日本語のコメントアウトをする場合jlistingが必要
\usepackage{color}
\usepackage{txfonts}
%ここからソースコードの表示に関する設定
\lstset{
  language=c++,
  basicstyle={\ttfamily},
  identifierstyle={\small},
  commentstyle={\textit},
  keywordstyle={\small\bfseries},
  ndkeywordstyle={\small},
  stringstyle={\small\ttfamily},
  frame={tb},
  breaklines=true,
  columns=[l]{fullflexible},
  numbers=left,
  xrightmargin=0zw,
  xleftmargin=3zw,
  numberstyle={\scriptsize},
  stepnumber=1,
  numbersep=1zw,
  lineskip=-0.5ex
}
%ここまでソースコードの表示に関する設定

\title{C++スニペット} 
\author{xryuseix}
\date{\today} 

\begin{document}

\maketitle

\pagenumbering{roman}
\tableofcontents%目次
\clearpage

\pagenumbering{arabic}


\color{white}
\section{BoyerMoore}
\color{black}
\begin{lstlisting}[caption=BoyerMoore]

class BoyerMoore {
  public:
  string text;
  string pattern;
  int n;
  int m;
  map<char, int> lambda;
  BoyerMoore(string text_, string pattern_) : 
    text(text_), pattern(pattern_), n(text_.size()), m(pattern_.size()) {
    compute_lambda();
  };
  void compute_lambda(void) {
    for(int j = 1; j <= m; j++) {
      lambda[pattern.at(j - 1)] = j;
    }
  };
  int get_lambda(const char& c) {
    if (lambda.find(c) != lambda.end()) {
      return lambda[c];
    } else {
      return 0;
    }
  };
  bool match(void) {
    int s = 0;
    while(s <= n - m) {
      int j = m;
      while(j > 0 && pattern.at(j - 1) == text.at(s + j - 1)) {
        j--;
      }
      if(j == 0) {
        return true;//ここを消すとsが文字列の位置を示す
        s++;
      } else {
        s += std::max(1, j - get_lambda(text.at(s + j - 1)));
      }
    }
    return false;
  };
};

\end{lstlisting}

\color{white}
\section{BridgeArticulation}
\color{black}
\begin{lstlisting}[caption=BridgeArticulation]

class BridgeArticulation {
  int N, num = 0;
  vvi G;
  vi pre, low;
  vb isPassed;

  int culcLow(const int v, const int bef) {
    int nowLow = num;
    low[v] = pre[v] = nowLow;
    for(auto ne : G[v]) {
      if(ne == bef) continue;
      if(ne == 0) {
        low[0] = -1;
      }
      if(pre[ne] == -1) {
        num++;
        culcLow(ne, v);
      }
      chmin(nowLow, low[ne]);
    }
    return low[v] = nowLow;
  }

  void traceGraph(const int v, const int bef) {
    bool is_articulation = false;
    for(auto ne : G[v]) {
      if(ne == bef) continue;
      if(!isPassed[ne]) {
        if(low[ne] >= pre[v] && (bef != -1 || G[v].size() >= 2)) {
          is_articulation = true;
        }
        if(low[ne] == pre[ne]) {
          bridges.emplace_back(min(v, ne), max(v, ne));
        }
        isPassed[ne] = true;
        traceGraph(ne, v);
      }
    }
    if(is_articulation) {
      articulation.push_back(v);
    }
  }

public:
  vpii bridges; // 橋
  vi articulation; // 関節点

  BridgeArticulation(const int _n, const vvi _G) : N(_n), G(_G) {
    pre = vi(N, -1);
    low = vi(N, INF);
    isPassed = vb(N, false);
    isPassed[0] = true;
  }

  void findBridges() {
    culcLow(0, -1);
    traceGraph(0, -1);
    Sort(bridges);
  }

  void show() {
    for(auto p : bridges) {
      cout << p.fi << \" \" << p.se << endl; 
    }
  }
};

\end{lstlisting}

\color{white}
\section{CoordinateCompression}
\color{black}
\begin{lstlisting}[caption=CoordinateCompression]

class Compress {
public:
  int before_W, before_H, N;
  vi before_X1, before_X2, before_Y1, before_Y2;
  int after_W, after_H;
  vi after_X1, after_X2, after_Y1, after_Y2;
  
  // (x1,y1) -> (x2, y2) の直線上のマスが塗られているとする
  // 点の場合は (x1,y1) == (x2, y2) とする
  // 四角形の場合は直線の集合とする
  Compress(int max_h, int max_w, int n, vi x1, vi x2, vi y1, vi y2) {
    before_H = max_h;
    before_W = max_w;
    N = n;
    before_X1 = x1;
    before_X2 = x2;
    before_Y1 = y1;
    before_Y2 = y2;
    after_X1 = vi(max_w);
    after_X2 = vi(max_w);
    after_Y1 = vi(max_h);
    after_Y2 = vi(max_h);
  }

  void compress(void) {
    after_W = exec_compress(before_X1, before_X2, after_X1, after_X2, before_W, \"width\");
    after_H = exec_compress(before_Y1, before_Y2, after_Y1, after_Y2, before_H, \"height\");
  }

  void before_show(void) {
    vvc v(before_H, vc(before_W, '_'));
    cout << \"H = \" << before_H << \" W = \" << before_W << endl;
    for(int i = 0; i < N; i++) {
      for(int y = before_Y1[i]; y <= before_Y2[i]; y++) {
        for(int x = before_X1[i]; x <= before_X2[i]; x++) {
          v[y][x] = '#';
        }
      }
    }
    rep(i, before_H){
      rep(j, before_W){
        cout << v[i][j];
      }
      cout<<endl;
    }
    cout << endl;
  }

  void after_show(void) {
    vvc v(after_H, vc(after_W, '_'));
    cout << \"H = \" << after_H << \" W = \" << after_W << endl;
    for(int i = 0; i < N; i++) {
      for(int y = after_Y1[i]; y <= after_Y2[i]; y++) {
        for(int x = after_X1[i]; x <= after_X2[i]; x++) {
          v[y][x] = '#';
        }
      }
    }
    rep(i, after_H){
      rep(j, after_W){
        cout << v[i][j];
      }
      cout<<endl;
    }
    cout << endl;
  }

private:
  int exec_compress(vi &z1, vi &z2, vi &aft_z1, vi &aft_z2, int max_len, string mode) {
    vector<int> zs;
    for(int i = 0; i < N; i++) {
      if(z1[i] > z2[i]) swap(z1[i], z2[i]);

      zs.push_back(z1[i]);
      zs.push_back(z2[i]);

      if(mode == \"width\") {
        if(z2[i] + 1 <= max_len) zs.push_back(z2[i] + 1);
      } else if(mode == \"height\") {
        if(0 < z1[i] - 1) zs.push_back(z1[i] - 1);
      }
    }
    zs.push_back(1);
    zs.push_back(max_len);

    sort(zs.begin(), zs.end());
    zs.erase(unique(zs.begin(), zs.end()), zs.end());

    for(int i = 0; i < N; i++) {
      aft_z1[i] = find(zs.begin(), zs.end(), z1[i]) - zs.begin() + 1;
      aft_z2[i] = find(zs.begin(), zs.end(), z2[i]) - zs.begin() + 1;
    }
    return zs.size();
  }
};

\end{lstlisting}

\color{white}
\section{LazySegmentTree}
\color{black}
\begin{lstlisting}[caption=LazySegmentTree]

template <typename T>
class Sum {
public:
  // 単位元
  T unit;
  
  Sum(void) {
    // 単位元
    unit = 0;
  }

  // 演算関数
  T calc(T a, T b) {
    return a + b; 
  }
};

template <typename T, class MONOID>
class LazySegmentTree {
public:
  // セグメント木の葉の要素数
  int n;

  // セグメント木
  vector<T> tree, lazy;

  // モノイド
  MONOID mono;

  LazySegmentTree(vector<T> &v) {
    n = 1 << (int)ceil(log2(v.size()));
    tree = vector<T>(n << 1);
    lazy = vector<T>(n << 1, mono.unit);
    for(int i = 0; i < v.size(); ++i) {
      update(i, v[i]);
    }
    for(int i = v.size(); i < n; ++i) {
      update(i, mono.unit);
    }
  }

  // k番目の値(0-indexed)をxに変更
  void update(int k, T x) {
    k += n;
    tree[k] = x;
    for(k = k >> 1; k > 0; k >>= 1){
      tree[k] = mono.calc(tree[k << 1 | 0], tree[k << 1 | 1]);
    }
  }

  // [l, r)の最小値(0-indexed)を求める.
  T query(int l, int r) {
    T res = mono.unit;
    l += n;
    r += n;
    while(l < r) {
      if(l & 1) {
        res = mono.calc(res, tree[l++]);
      }
      if(r & 1) {
        res = mono.calc(res, tree[--r]);
      }
      l >>= 1;
      r >>= 1;
    }
    return res;
  }

  // k番目のノードの遅延評価を行う
  void eval(int k, int l, int r) {
    // 遅延評価配列が空でない時,値を伝播する
    if(lazy[k] != mono.unit) {
      tree[k] += lazy[k];
      if(r - l > 1) {
        lazy[k<<1|0] += lazy[k]>>1;
        lazy[k<<1|1] += lazy[k]>>1;
      }

      // 伝播が終わったので自ノードの遅延配列を空にする
      lazy[k] = mono.unit;
    }
  }

  // 区間[l, r)にxを足す(遅延評価)
  void add(int l, int r, ll x) {
    add(l, r, x, 1, 0, n);
  }

  void add(int a, int b, ll x, int k, int l, int r) {
    // k番目のノードに対して遅延評価を行う
    eval(k, l, r);

    // 範囲外なら何もしない
    if(b <= l || r <= a) return;
    
    // 完全に被覆しているならば、遅延配列に値を入れた後に評価
    if(a <= l && r <= b) {
      lazy[k] += (r - l) * x;
      eval(k, l, r);
    } else {
      add(a, b, x, k<<1|0, l, (l + r)>>1);
      add(a, b, x, k<<1|1, (l + r)>>1, r);
      tree[k] = tree[k<<1|0] + tree[k<<1|1];
    }
  }

  // 区間[l, r)の合計を取得する
  ll getRange(int l, int r) {
    return getRange(l, r, 1, 0, n);
  }

  ll getRange(int a, int b, int k, int l, int r) {
    if(b <= l || r <= a) return mono.unit;
    // 関数が呼び出されたら評価!
    eval(k, l, r);
    if(a <= l && r <= b) return tree[k];
    ll vl = getRange(a, b, k<<1|0, l, (l + r)>>1);
    ll vr = getRange(a, b, k<<1|1, (l + r)>>1, r);
    return vl + vr;
  }

  T operator[](int k) {
    // st[i]で添字iの要素の値を返す
    if(k - n >= 0 || k < 0) {
      return -INF;
    }
    return tree[tree.size() - n + k];
  }

  void show(void) {
    showTree();
    showLazy();
  }

  void showTree(void) {
    int ret = 2;
    for(int i = 1; i < 2*n; ++i) {
      if(tree[i] == mono.unit) cout << \"UNIT \";
      else cout << tree[i] << \" \";
      if(i == ret - 1) {
        cout << endl;
        ret <<= 1;
      }
    }
    cout << endl;
  }

  void showLazy(void) {
    int ret = 2;
    for(int i = 1; i < 2*n; ++i) {
      if(lazy[i] == mono.unit) cout << \"UNIT \";
      else cout << lazy[i] << \" \";
      if(i == ret - 1) {
        cout << endl;
        ret <<= 1;
      }
    }
    cout << endl;
  }
};

\end{lstlisting}

\color{white}
\section{MaximumFlow}
\color{black}
\begin{lstlisting}[caption=MaximumFlow]

class MaximumFlow {

  int v;

  // 辺を表す構造体(行き先,容量,逆辺のインデックス)
  struct edge {
    int to;
    int cap;
    int rev;
  };

  vector<vector<edge>> G; // グラフの隣接リスト表現
  vector<bool> used; // DFSですでに調べたかのフラグ

  // 増加パスをDFSで探す(今いる頂点, ゴールの頂点, 今の頂点以降のフローの最小値)
  int dfs(int v, int t, int f) {
    if (v == t) return f;
    used[v] = true;
    for (int i = 0; i < G[v].size(); i++) {
      // vから行ける&&cap>0の頂点を全てたどる
      edge& e = G[v][i];
      if (!used[e.to] && e.cap > 0) {
        // 次の頂点(e.to)以降でtまで行けるパスを探索し,その時のフローの最小値をdとする
        int d = dfs(e.to, t, min(f, e.cap));
        if (d > 0) {
          e.cap -= d;
          G[e.to][e.rev].cap += d;
          return d;
        }
      }
    }
    return 0;
  }

public:
  MaximumFlow(int _v) : v(_v) {
    used = vector<bool>(v);
    G = vector<vector<edge>>(v);
  }

  // fromからtoへ向かう容景capの辺をグラフに追加する
  void add(int from, int to, int cap) {
    G[from].push_back((edge){to, cap, (int)G[to].size()});
    G[to].push_back((edge){from, 0, (int)G[from].size() - 1});
  }

  // sからtへの最大流を求める
  int maxFlow(int s, int t) {
    int flow = 0;
    while (true) {
      used = vector<bool>(v);
      int f = dfs(s, t, INF);
      if (f == 0) {
        return flow;
      }
      flow += f;
    }
  }
};

\end{lstlisting}

\color{white}
\section{MinCostFlow}
\color{black}
\begin{lstlisting}[caption=MinCostFlow]

class MinCostFlow {

  int V; // 頂点数

  // 辺を表す構造体(行き先,容量,逆辺のインデックス)
  struct edge {
    int to;
    int cap;
    int cost;
    int rev;
  };

  vector<vector<edge>> G; // グラフの隣接リスト表現
  vector<int> h; // ポテンシャル
  vector<int> prevV; // 直前の頂点
  vector<int> prevE; // 直前の辺
  vector<int> dist; // 最短距離
  typedef pair<int, int> PI;

public:
  MinCostFlow(int _v) : V(_v) {
    G = vector<vector<edge>>(V);
    h = vector<int>(V);
    prevV = vector<int>(V);
    prevE = vector<int>(V);
  }

  // fromからtoへ向かう容景capの辺をグラフに追加する
  void add(int from, int to, int cap, int cost) {
    G[from].push_back((edge){to, cap, cost, (int)G[to].size()});
    G[to].push_back((edge){from, 0, -cost, (int)G[from].size() - 1});
  }

  // sからtへの最大流を求める
  int minCostFlow(int s, int t, int f) {
    int res = 0;
    while (f > 0) {
      // ダイクストラ法を用いてhを更新する
      priority_queue<PI, vector<PI>, greater<PI>> que;
      dist = vector<int>(V, INF);
      dist[s] = 0;
      que.push(PI(0, s));
      while(!que.empty()) {
        PI p = que.top();
        que.pop();
        int v = p.second;
        if(dist[v] < p.first) continue;
        for (int i = 0; i < G[v].size(); i++) {
          edge e = G[v][i];
          if(e.cap > 0 && dist[e.to] > dist[v] + e.cost + h[v] - h[e.to]) {
            dist[e.to] = dist[v] + e.cost + h[v] - h[e.to];
            prevV[e.to] = v;
            prevE[e.to] = i;
            que.push(PI(dist[e.to], e.to));
          }
        }
      }
      if(dist[t] == INF) {
        // これ以上流せない
        return -1;
      }
      for(int v = 0; v < V; v++) {
        h[v] += dist[v];
      }
      // s-t問最短路に沿って目一杯流す
      int d = f;
      for(int v = t; v != s; v = prevV[v] ) {
        d = min(d, G[prevV[v]][prevE[v]].cap);
      }
      f -= d;
      res += d*h[t];
      for(int v = t; v != s; v = prevV[v]) {
        edge& e = G[prevV[v]][prevE[v]];
        e.cap -= d;
        G[v][e.rev].cap += d;
      }
    }
    return res;
  }
};

\end{lstlisting}

\color{white}
\section{RollingHash}
\color{black}
\begin{lstlisting}[caption=RollingHash]

std::mt19937 mt{ std::random_device{}() };
std::uniform_int_distribution<int> dist(129, INF);
const int BASE = dist(mt);

class RollingHash {
public:
  string str;
  vector<ull> powBase, csumHash;
  const ull ROLMOD = (1LL << 61) - 1;
  const ull MASK30 = (1LL << 30) - 1;
  const ull MASK31 = (1LL << 31) - 1;
  const ull LLMAX = ROLMOD*((1LL << 3) - 1);

  RollingHash(const string s) : str(s) {
    powBase.resize(s.size() + 1);
    csumHash.resize(s.size() + 1);
    powBase[0] = 1;
    for(int i = 0; i < s.size(); i++) {
      powBase[i + 1] = calcMod(multiple(powBase[i], BASE));
    }
  }

  void rollingHash() {
    csumHash[0] = 0;
    for(int i = 0; i < str.size(); ++i) {
      csumHash[i + 1] = calcMod(multiple(csumHash[i], BASE) + str[i]);
    }
  }

  ull getHash(const int begin, const int length) {
    return calcMod(csumHash[begin + length] + LLMAX - multiple(csumHash[begin], powBase[length]));
  }

  string substr(const int begin) {
    return str.substr(begin);
  }

  string substr(const int begin, const int length) {
    if(length < 0) {
      return str.substr(begin, str.size() + length - begin + 1);
    } else {
      return str.substr(begin, length);
    }
  }

private:
  ull calcMod(const ull num) {
    const ull modNum = (num & ROLMOD) + (num >> 61);
    return (modNum >= ROLMOD) ? modNum - ROLMOD : modNum;
  }

  ull multiple(const ull leftNum, const ull rightNum) {
    ull lu = leftNum >> 31;
    ull ld = leftNum & MASK31;
    ull ru = rightNum >> 31;
    ull rd = rightNum & MASK31;
    ull middleBit = ld * ru + lu * rd;
    return ((lu * ru) << 1) + ld * rd + ((middleBit & MASK30) << 31) + (middleBit >> 30);
  }

};

\end{lstlisting}

\color{white}
\section{SegmentTree}
\color{black}
\begin{lstlisting}[caption=SegmentTree]

template <typename T>
class Sum {
public:
  // 単位元
  T unit;

  Sum(void) {
    // 単位元
    unit = 0;
  }

  // 演算関数
  T calc(T a, T b) { return a + b; }
};

template <typename T>
struct Min {
public:
  // 単位元
  T unit;

  Min(void) {
    // 単位元
    unit = INF;
  }

  // 演算関数
  T calc(T a, T b) { return min(a, b); }
};

template <typename T, class MONOID>
class SegmentTree {
public:
  // セグメント木の葉の要素数
  int n;

  // セグメント木
  vector<T> tree;

  // モノイド
  MONOID mono;

  SegmentTree(vector<T>& v)
    : n(1 << (int)ceil(log2(v.size()))),
    tree(vector<T>(n << 1, mono.unit)) {
    for (int i = 0; i < v.size(); ++i) {
      tree[i + n] = v[i];
    }
    for (int i = n - 1; i > 0; --i) {
      tree[i] = mono.calc(tree[i << 1 | 0], tree[i << 1 | 1]);
    }
  }

  SegmentTree(int _n)
    : n(1 << (int)ceil(log2(_n))), tree(vector<T>(n << 1, mono.unit)) {}

  // k番目の値(0-indexed)をxに変更
  void update(int k, T x) {
    k += n;
    tree[k] = x;
    for (k = k >> 1; k > 0; k >>= 1) {
      tree[k] = mono.calc(tree[k << 1 | 0], tree[k << 1 | 1]);
    }
  }

  // k番目の値(0-indexed)をxを加算
  void add(int k, T x) {
    k += n;
    tree[k] += x;
    for (k = k >> 1; k > 0; k >>= 1) {
      tree[k] = mono.calc(tree[k << 1 | 0], tree[k << 1 | 1]);
    }
  }

  // [l, r)の最小値(0-indexed)を求める.
  T query(int l, int r) {
    T res = mono.unit;
    l += n;
    r += n;
    while (l < r) {
      if (l & 1) {
        res = mono.calc(res, tree[l++]);
      }
      if (r & 1) {
        res = mono.calc(res, tree[--r]);
      }
      l >>= 1;
      r >>= 1;
    }
    return res;
  }

  T operator[](int k) {
    // st[i]で添字iの要素の値を返す
    if (k - n >= 0 || k < 0) {
      return -INF;
    }
    return tree[tree.size() - n + k];
  }

  void show(int elements = 31) {
    int ret = 2;
    for (int i = 1; i < max(2 * n, elements); ++i) {
      if (tree[i] == mono.unit)
        cout << \"UNIT \";
      else
        cout << tree[i] << \" \";
      if (i == ret - 1) {
        cout << endl;
        ret <<= 1;
      }
    }
    cout << endl;
  }
};

\end{lstlisting}

\color{white}
\section{StronglyConnectedComponent}
\color{black}
\begin{lstlisting}[caption=StronglyConnectedComponent]

class StronglyConnectedComponent {
public:
  int V;  // 頂点数
  int SubGraph;   // 強連結成分の数
  vvi Graph;  // グラフの隣接リスト表現
  vvi revGraph;   // 辺の向きを逆にしたグラフ
  vvi SmallGraph; // 強連結成分分解によって縮めたグラフ
  vi dfsline; // 帰りがけ順の並び
  vi compo;   // cmp[i]で頂点iの属するグループ
  vb used;// すでに調べたか

  StronglyConnectedComponent(int v) {
    V = v;
    Graph = vvi(v);
    revGraph = vvi(v);
    used = vb(v);
    compo = vi(v);
  }

  int operator[](int k) {
    // scc[i]でi番目の頂点のグループ番号を返す
    return compo[k];
  }

  void add_edge(int from, int to) {
    Graph[from].push_back(to);
    revGraph[to].push_back(from);
  }

  void dfs(int v) {
    used[v] = true;
    for(int i = 0; i < Graph[v].size(); i++) {
      if(!used[Graph[v][i]]) dfs(Graph[v][i]);
    }
    dfsline.push_back(v);
  }

  void revdfs(int v, int k) {
    used[v] = true;
    compo[v] = k;
    for(int i = 0; i < revGraph[v].size(); i++) {
      if(!used[revGraph[v][i]]) revdfs(revGraph[v][i], k);
    }
  }

  int scc(void) {
    used = vb((int)used.size(), false);
    dfsline.clear();
    for(int v = 0; v < V; v++) {
      if(!used[v]) dfs(v);
    }
    used = vb(used.size(), false);
    SubGraph = 0;
    for(int i = dfsline.size() - 1; i >= 0; i--) {
      if(!used[dfsline[i]]) revdfs(dfsline[i], SubGraph++);
    }
    for(int i = 0; i < compo.size(); i++) {
      compo[i] = SubGraph - compo[i] - 1;
    }
    return SubGraph;
  }

  void build(void) {
    // 縮めたグラフを構築する
    SmallGraph = vvi(SubGraph);
    for (int i = 0; i < Graph.size(); i++) {
      for(int j = 0; j < Graph[i].size(); j++) {
        int to = Graph[i][j];
        int s = compo[i], t = compo[to];
        if (s != t){
          SmallGraph[s].push_back(t);
        }
      }
    }
    for(int i = 0; i < SmallGraph.size(); i++) {
      // 被った辺を削除
      SmallGraph[i].erase(unique(SmallGraph[i].begin(), SmallGraph[i].end()), SmallGraph[i].end());
    }
  }

  void show_set_to_edge(void) {
    for(int i = 0; i < SmallGraph.size(); i++) {
      cout << \"集合\" << i << \"から出ている辺 : \";
      for(int j = 0; j < SmallGraph[i].size(); j++) {
        cout << SmallGraph[i][j] << ' ';
      }
      cout << endl;
    }
    cout << endl;
  }

  void show_group_of_node(void) {
    for(int i = 0; i < V; i++) {
      cout << \"頂点\" << i << \"の属するグループ : \" << compo[i] << endl;
    }
    cout << endl;
  }

  void show_node_in_group(void) {
    vvi group(SubGraph);
    for(int i = 0; i < compo.size(); i++) {
      group[compo[i]].push_back(i);
    }
    for(int i = 0; i < SmallGraph.size(); i++) {
      cout << \"グループ\" << i << \"に属する頂点 : \";
      for(int j = 0; j < group[i].size(); j++) {
        cout << group[i][j] << ' ';
      }
      cout << endl;
    }
    cout << endl;
  }
};

\end{lstlisting}

\color{white}
\section{Trie}
\color{black}
\begin{lstlisting}[caption=Trie]

template <int char_size, int base>
class Trie {
public:
  struct Node {
    vector<int> next;// 子の頂点番号を格納。存在しなければ-1
    vector<int> accept;  // 末端がこの頂点になる文字列(受理状態)のID
    int c;   // 文字,baseからの距離
    int common;  // いくつの文字列がこの頂点を共有しているか
    Node(int c_) : c(c_), common(0) { next = vector<int>(char_size, -1); }
  };
  vector<Node> nodes;  // trie 木本体
  Trie() { nodes.push_back(Node(0)); }

  /*
  単語の検索
  prefix ... 先頭のwordが一致していればいいかどうか
  */
  bool search(const string word, const bool prefix = false) {
    int node_id = 0;
    for (auto w : word) {
      int c = w - base;
      int next_id = nodes[node_id].next[c];
      // printf(\"%c %d %d\n\"w,c,next_id);
      if (next_id == -1) {
        return false;
      }
      node_id = next_id;
    }
    return (prefix) ? true : nodes[node_id].accept.size();
  }

  /*
  単語の挿入
  word_id ... 何番目に追加する単語か
  */
  void insert(const string &word) {
    int node_id = 0;
    for (auto w : word) {
      int c = w - base;
      int next_id = nodes[node_id].next[c];
      if (next_id == -1) {  // 次の頂点が存在しなければ追加
        next_id = nodes.size();
        nodes.push_back(Node(c));
        nodes[node_id].next[c] = next_id;
      }
      ++nodes[node_id].common;
      node_id = next_id;
    }
    ++nodes[node_id].common;
    nodes[node_id].accept.push_back(nodes[0].common - 1); // 単語の終端なので、頂点に番号を入れておく
  }

  /*
  prefixの検索
  */
  bool start_with(const string& prefix) { return search(prefix, true); }

  /*
  挿入した単語数
  */
  int count() const { return nodes[0].common; }

  /*
  頂点数
  */
  int size() const { return nodes.size(); }
};

\end{lstlisting}

\color{white}
\section{bellmanford}
\color{black}
\begin{lstlisting}[caption=bellmanford]

// 頂点fromから頂点toへのコストcostの辺
struct bf_edge {
  int from;
  int to;
  int cost;
};

class Bellman_Ford{
public:
  vector<bf_edge> es; // 辺
  vector<int> d; // d[i]...頂点sから頂点iまでの最短距離
  int V, E; // Vは頂点数、Eは辺数

  Bellman_Ford(int v, int e) {
    V = v;
    E = e;
    d = vector<int>(v);
  }

  void add(int from, int to, int cost) {
    bf_edge ed = {from, to, cost};
    es.push_back(ed);
  }

  // s番目の頂点から各頂点への最短距離を求める
  // trueなら負の閉路が存在する
  bool shortest_path(int s) {
    for(int i = 0; i < V; i++) {
      d[i] = 0;
    }
    for (int i = 0; i < 3*V; i++) {
      for(int j = 0; j < E; j++) {
        bf_edge e = es[j];
        if(d[e.to] > d[e.from] + e.cost) {
          d[e.to] = d[e.from] + e.cost;

          // 3n回目にも更新があるなら負の閉路が存在する
          if(i == V - 1)return true;
        }
      }
    }
    return false;
  }
};

\end{lstlisting}

\color{white}
\section{bfs}
\color{black}
\begin{lstlisting}[caption=bfs]

// 各座標に格納したい情報を構造体にする
// 今回はX座標,Y座標,深さ(距離)を記述している
struct Corr {
  int x;
  int y;
  int depth;
};
queue<Corr> q;
int bfs(vector<vector<int>> grid) {
  // 既に探索の場所を1,探索していなかったら0を格納する配列
  vector<vector<int>> ispassed(grid.size(), vector<int>(grid[0].size(), false));
  // このような記述をしておくと,この後のfor文が綺麗にかける
  int dx[8] = {1, 0, -1, 0};
  int dy[8] = {0, 1, 0, -1};
  while(!q.empty()) {
    Corr now = q.front();q.pop();
    /*
      今いる座標は(x,y)=(now.x, now.y)で,深さ(距離)はnow.depthである
      ここで,今いる座標がゴール(探索対象)なのか判定する
    */
    for(int i = 0; i < 4; i++) {
      int nextx = now.x + dx[i];
      int nexty = now.y + dy[i];

      // 次に探索する場所のX座標がはみ出した時
      if(nextx >= grid[0].size()) continue;
      if(nextx < 0) continue;

      // 次に探索する場所のY座標がはみ出した時
      if(nexty >= grid.size()) continue;
      if(nexty < 0) continue;

      // 次に探索する場所が既に探索済みの場合
      if(ispassed[nexty][nextx]) continue;

      ispassed[nexty][nextx] = true;
      Corr next = {nextx, nexty, now.depth+1};
      q.push(next);
    }
  }
}

\end{lstlisting}

\color{white}
\section{binarysearch}
\color{black}
\begin{lstlisting}[caption=binarysearch]

// vector vの中のn以下の数で最大のものを返す
int bs(vector<ll> &v, ll x){
  int ok = -1; //これがx以下 
  int ng = v.size(); //x以上 
  // 問題によってokとngを入れ替える
  while(abs(ok - ng) > 1){
    int mid = (ok + ng)/2;
    if(v[mid] <= x) ok = mid;
    else ng = mid;
  }
  return ok;
}

\end{lstlisting}

\color{white}
\section{bit}
\color{black}
\begin{lstlisting}[caption=bit]

template <typename T>
class BIT {
  int N;
  vector<T> tree;

public:
  BIT(vector<T>& v) : N(v.size()), tree(vector<T>(v.size() + 1)) {
    rep(i, v.size()) { add(i, v[i]); }
  }
  BIT(int n) : N(n), tree(vector<T>(n + 1)) {}

  void add(int i, T x = 1) {
    for (++i; i <= N; i += i & -i) {
      tree[i] += x;
    }
  }

  T sum(int l) {  // [0, l)の和
    T x = 0;
    for (; l; l -= l & -l) {
      x += tree[l];
    }
    return x;
  }

  T sum(int l, int r) {  // [l, r)の和
    return sum(r) - sum(l);
  }

  T sum_back(int l) {  // [l, N)の和
    return sum(N) - sum(l);
  }
};

\end{lstlisting}

\color{white}
\section{boostlibrary}
\color{black}
\begin{lstlisting}[caption=boostlibrary]

#include <boost/multiprecision/cpp_dec_float.hpp>
#include <boost/multiprecision/cpp_int.hpp>
namespace mp = boost::multiprecision;
// 任意長整数型
using Bint = mp::cpp_int;
// 仮数部が10進数で1024桁の浮動小数点数型(TLEしたら小さくする)
using Real = mp::number<mp::cpp_dec_float<1024>>;

\end{lstlisting}

\color{white}
\section{codeforces}
\color{black}
\begin{lstlisting}[caption=codeforces]

int Q;
cin >> Q;
while(Q--) {
  
}

\end{lstlisting}

\color{white}
\section{combination}
\color{black}
\begin{lstlisting}[caption=combination]

#define MAX_NCK 201010
ll f[MAX_NCK], rf[MAX_NCK];

// modinvも呼ぶ!!

bool isinit = false;

void init(void) {
  f[0] = 1;
  rf[0] = modinv(1);
  for(int i = 1; i < MAX_NCK; i++) {
    f[i] = (f[i - 1] * i) % MOD;
    rf[i] = modinv(f[i]);
  }
}

ll nCk(int n, int k) {
  if(!isinit) {
    init();
    isinit = 1;
  }
  if(n < k) return 0;
  ll nl = f[n]; // n!
  ll nkl = rf[n - k]; // (n-k)!
  ll kl = rf[k]; // k!
  ll nkk = (nkl * kl) % MOD;

  return (nl * nkk) % MOD;
}

\end{lstlisting}

\color{white}
\section{combination2}
\color{black}
\begin{lstlisting}[caption=combination2]

// 逆元を使わないnCk
// なんでこうなるかわよくわからん
const int NCKMAX = 3;
int C[NCKMAX][NCKMAX];
bool isinit = false;
void init() {
  rep(i, NCKMAX) { C[i][0] = C[i][i] = 1; }
  for (int i = 1; i < NCKMAX; i++) {
    for (int j = 1; j < i; j++) {
      C[i][j] = (C[i - 1][j - 1] + C[i - 1][j]) % NCKMAX;
    }
  }
}

int nCk(int n, int k) {
  if (!isinit) {
    init();
    isinit = 1;
  }
  if (k < 0 || k > n) return 0;
  int ans = 1;
  while (n > 0) {
    int x = n % NCKMAX;
    int y = k % NCKMAX;
    n /= NCKMAX;
    k /= NCKMAX;
    ans = (ans * C[x][y]) % NCKMAX;
  }
  return ans;
}

\end{lstlisting}

\color{white}
\section{conlis}
\color{black}
\begin{lstlisting}[caption=conlis]

int conlis(vector<int>& v) {
  vi dp(v.size() + 1, 0);
  int ans = 0;
  for(int i = 0; i < v.size(); i++) {
    dp[v[i]] = dp[v[i] - 1] + 1;
    ans = max(ans, dp[v[i]]);
  }
  return ans;
}

\end{lstlisting}

\color{white}
\section{digitsum}
\color{black}
\begin{lstlisting}[caption=digitsum]

int digsum(int n) {
  int res = 0;
  while(n > 0) {
    res += n%10;
    n /= 10;
  }
  return res;
}

\end{lstlisting}

\color{white}
\section{dijkstra}
\color{black}
\begin{lstlisting}[caption=dijkstra]

class DIJKSTRA {
public:
  int V;

  struct dk_edge {
    int to;
    ll cost;
  };

  typedef pair<ll, int> PI;  // firstは最短距離、secondは頂点の番号
  vector<vector<dk_edge> > G;
  vector<ll> d;  //これ答え。d[i]:=V[i]までの最短距離
  vector<int> prev;  //経路復元

  DIJKSTRA(int size) {
    V = size;
    G = vector<vector<dk_edge> >(V);
    prev = vector<int>(V, -1);
  }

  void add(int from, int to, ll cost) {
    dk_edge e = {to, cost};
    G[from].push_back(e);
  }

  void dijkstra(int s) {
    // greater<P>を指定することでfirstが小さい順に取り出せるようにする
    priority_queue<PI, vector<PI>, greater<PI> > que;
    d = vector<ll>(V, LLINF);
    d[s] = 0;
    que.push(PI(0, s));

    while (!que.empty()) {
      PI p = que.top();
      que.pop();
      int v = p.second;
      if (d[v] < p.first) continue;
      for (int i = 0; i < G[v].size(); i++) {
        dk_edge e = G[v][i];
        if (d[e.to] > d[v] + e.cost) {
          d[e.to] = d[v] + e.cost;
          prev[e.to] = v;
          que.push(PI(d[e.to], e.to));
        }
      }
    }
  }
  vector<int> get_path(int t) {
    vector<int> path;
    for (; t != -1; t = prev[t]) {
      // tがsになるまでprev[t]をたどっていく
      path.push_back(t);
    }
    //このままだとt->sの順になっているので逆順にする
    reverse(path.begin(), path.end());
    return path;
  }
  void show(void) {
    for (int i = 0; i < d.size() - 1; i++) {
      cout << d[i] << \" \";
    }
    cout << d[d.size() - 1] << endl;
  }
};

\end{lstlisting}

\color{white}
\section{eratosthenes}
\color{black}
\begin{lstlisting}[caption=eratosthenes]

class Sieve{
  int N;
  void eratosmake(void) {
    // iを残してiの倍数を消していく
    for(int i = 2; i < N; i++) {
      if(nums[i] == 1) {
        for(int j = i + i; j < N; j += i){
          nums[j] = i;
        }
      }
    }
  }

public:
  vector<int> nums;
  Sieve(int n):N(n){
    nums = vi(n+1, 1);
    eratosmake();
  }
  bool isPrime(int n) {
    return nums[n] == 1;
  }
  int minPrimeFactor(int n) {
    return nums[n];
  }
};

\end{lstlisting}

\color{white}
\section{extgcd}
\color{black}
\begin{lstlisting}[caption=extgcd]

// x,y に ax + by = gcd(a, b) を満たす値が格納される
// 返り値は<gcd(a, b), x, y>
// auto [g, x, y] = extgcd(a, b);のように呼び出す
tuple<ll, ll, ll> extgcd(ll a, ll b) {
  if (b == 0) {
    return {a, 1, 0};
  }
  auto [g, x, y] = extgcd(b, a % b);
  return {g, y, x - a / b * y};
}

\end{lstlisting}

\color{white}
\section{gcd}
\color{black}
\begin{lstlisting}[caption=gcd]

ll gcd(ll a, ll b) { return b ? gcd(b, a%b) : a;}

\end{lstlisting}

\color{white}
\section{getline}
\color{black}
\begin{lstlisting}[caption=getline]

string getline() {
  string s;
  fflush(stdin);
  getline(cin, s);
  return s;
}

\end{lstlisting}

\color{white}
\section{gpriorityqueue}
\color{black}
\begin{lstlisting}[caption=gpriorityqueue]

priority_queue<int, vector<int>, greater<int> > queue;

\end{lstlisting}

\color{white}
\section{gyakugen}
\color{black}
\begin{lstlisting}[caption=gyakugen]

// (a/b)%P の場合は,(a%P)*modinv(b)%P とする
ll modinv(ll a) {
  ll b = MOD, u = 1, v = 0;
  while (b) {
    ll t = a / b;
    a -= t * b; swap(a, b);
    u -= t * v; swap(u, v);
  }
  u %= MOD;
  if (u < 0) u += MOD;
  return u;
}

\end{lstlisting}

\color{white}
\section{hakidashi}
\color{black}
\begin{lstlisting}[caption=hakidashi]

#define RANK 20 // 20元連立方程式まで解ける
/*
使用方法
  double a[RANK][RANK+1];
  int i, n;
  a[0][0] = 2; a[0][1] = 3; a[0][2] = 1; a[0][3] = 4;
  a[1][0] = 4; a[1][1] = 1; a[1][2] = -3 ; a[1][3] = -2;
  a[2][0] = -1; a[2][1] = 2; a[2][2] = 2; a[2][3] = 2;
  n = 3;
  hakidashi(a,n);
*/
void hakidashi(double a[][RANK+1], int n) { 
  double piv, t;
  int i, j, k;
  for (k = 0; k < n; k++) {
    piv = a[k][k];
    for (j = k; j < n + 1; j++) {
      a[k][j] = a[k][j]/piv;
    }
    for (i = 0; i < n; i++) {
      if (i != k) {
        t = a[i][k];
        for (j = k; j < n+1; j++) {
          a[i][j] = a[i][j] - t*a[k][j];
        }
      }
    }
  }
}

\end{lstlisting}

\color{white}
\section{hutei}
\color{black}
\begin{lstlisting}[caption=hutei]

void hutei(int a, int b, int c, bool minus) {
  vector<int> arr;

  // A / B = div...mod
  int A = max(a, b);
  int B = min(a, b);
  int div, mod;

  while(1) {
    div = A/B;
    mod = A%B;
    arr.push_back(div);

    A = B;
    B = mod;

    if(mod == 1) {
      break;
    }
  }

  vector<vector<int> > calc(2, vector<int> (arr.size() + 1, INF));

  for(int i = 0; i < arr.size() - 1; i++) {
    calc[0][i] = -arr[i];
  }
  calc[1][arr.size() - 1] = -arr[arr.size() - 1];
  calc[1][arr.size()] = 1;

  for(int i = arr.size()-2; i >= 0; i--) {
    calc[1][i] = calc[0][i]*calc[1][i + 1] + calc[1][i + 2];
  }

  int x = calc[1][0]*c;
  int y = calc[1][1]*c;

  if(minus) {
    y *= -1;
  }
  cout << a << \"(\" << b << \"m + \" << x << \")\";
  if(minus) {
    cout << \" - \";
  } else {
    cout << \" + \";
  }
  cout << b << \"(\" << a << \"m + \" << y << \")\" << \" = \" << c << endl;
}

\end{lstlisting}

\color{white}
\section{icpctemplate}
\color{black}
\begin{lstlisting}[caption=icpctemplate]

#include <iostream>
#include <algorithm>
#include <string>
#include <vector>
using namespace std;
typedef long long int ll;
typedef vector<int> vi;
#define rep(i,n) for(int i = 0; i < (n); ++i)
int main(void){}

\end{lstlisting}

\color{white}
\section{indexdistance}
\color{black}
\begin{lstlisting}[caption=indexdistance]

int indexdistance(vector<int> distance_array, char c) {
  return static_cast<int>(std::distance(std::begin(distance_array), std::find(std::begin(distance_array), std::end(distance_array), c)));
}

\end{lstlisting}

\color{white}
\section{intersect}
\color{black}
\begin{lstlisting}[caption=intersect]

void intersect(set<int> &Set_A, set<int> &Set_B, set<int> &Set_res) {
  set_intersection(Set_A.begin(), Set_A.end(), Set_B.begin(), Set_B.end(), inserter(Set_res, Set_res.end()));
}

\end{lstlisting}

\color{white}
\section{isPrime}
\color{black}
\begin{lstlisting}[caption=isPrime]

bool isPrime(ll x){
  if(x < 2)return 0;
  else if(x == 2) return 1;
  if(x%2 == 0) return 0;
  for(ll i = 3; i*i <= x; i += 2) if(x%i == 0) return 0;
  return 1;
}

\end{lstlisting}

\color{white}
\section{kika}
\color{black}
\begin{lstlisting}[caption=kika]

/* ==== 幾何ライブラリ ==== */
/* 点 */
struct Point {
  double x;
  double y;
  Point(double x = 0.0, double y = 0.0) : x(x), y(y) {}

  // === 四則演算の定義 ===
  friend inline Point operator + (const Point &p, const Point &q) {return Point(p.x + q.x, p.y + q.y);}
  friend inline Point operator - (const Point &p, const Point &q) {return Point(p.x - q.x, p.y - q.y);}
  friend inline Point operator * (const Point &p, const double a) {return Point(p.x * a, p.y * a);}
  friend inline Point operator * (const double a, const Point &p) {return Point(a * p.x, a * p.y);}
  friend inline Point operator * (const Point &p, const Point &q) {return Point(p.x * q.x - p.y * q.y, p.x * q.y + p.y * q.x);}
  friend inline Point operator / (const Point &p, const double a) {return Point(p.x / a, p.y / a);}

  // === その他の演算 ===
  // 反時計回りに90度回転
  friend Point rot90(const Point &p) {return Point(-p.y, p.x);}

  // 直線b,cからみて,aがどちら側にいるか判定
  // 1: bを上cを下とした時にaが右側にある, -1: aが左側にある, 0: aは直線bc上
  friend int simple_ccw(const Point &a, const Point &b, const Point &c) {
    if(OuterProduct(b-a, c-a) > EPS) return 1;
    if(OuterProduct(b-a, c-a) < -EPS) return -1;
    return 0;
  }

  // 内積
  friend inline double InnerProduct(const Point &p, const Point &q) {return p.x * q.x + p.y * q.y;}
  // 外積
  friend inline double OuterProduct(const Point &p, const Point &q) {return p.x * q.y - p.y * q.x;}

  // 二次元のノーム(ユークリッド距離)を計算
  friend inline double norm2(const Point &p) {return sqrt(InnerProduct(p, p));}

  // === 出力 ===
  friend ostream& operator << (ostream &s, const Point &p) {return s << '(' << p.x << \" \" << p.y << ')';}
};

/* 線 */
struct Line {
  vector<Point> line;

  Line(void) {}
  // 線分の時
  Line(Point a, Point b = Point(0.0, 0.0)) {
    // x座標が小さい方->y座標が小さい順にしておく
    if(a.x > b.x) {
      swap(a, b);
    } else if(a.x == b.x && a.y > b.y) {
      swap(a, b);
    }
    line.push_back(a);
    line.push_back(b);
  }
  // 多角形などの時
  Line(vector<Point> L) {
    /*
    // 基本はソートするとバグるのでしないこと
    sort(L.begin(), L.end(), [](Point const& lhs, Point const& rhs) {
      if(lhs.x != rhs.x) return lhs.x < rhs.x;
      else if(lhs.y != rhs.y) return lhs.y < rhs.y;
      return true;
    });
    */
    line = L;
  }

  // === 出力 ===
  friend ostream& operator << (ostream &s, const Line &l) {
    s << '{';
    rep(i, l.line.size()) {
      if(i) {
        s << \" \";
      }
      s << l.line[i];
    }
    s << '}';
    return s;
  }
};

/* 単位変換 */
double torad(int deg) {return (double)(deg) * MATHPI / 180;}
double todeg(double ang) {return ang * 180 / MATHPI;}

/* 直線や多角形の交点 */
Line crosspoint(const Line &L, const Line &M) {
  Line res;
  Line l = L;
  Line m = M;
  l.line.push_back(l.line[0]);
  m.line.push_back(m.line[0]);
  for(int i = 0; i < l.line.size() - 1; i++) {
    for(int j = 0; j < m.line.size() - 1; j++) {
      double d = OuterProduct(m.line[j + 1] - m.line[j], l.line[i + 1] - l.line[i]);
      if(abs(d) < EPS) continue;
      res.line.push_back(l.line[i] + (l.line[i + 1] - l.line[i]) * OuterProduct(m.line[j + 1] - m.line[j], m.line[j + 1] - l.line[i]) / d);
    }
  }
  return res;
}

/* 外心 */
Point gaisin(const Point a, const Point b, const Point c) {
  // 外心は三角形の二つの辺の垂直二等分線の交点
  Line ab( (a + b)/2, (a + b)/2 + rot90(a - b) );
  Line bc( (b + c)/2, (b + c)/2 + rot90(b - c) );
  return crosspoint(ab, bc).line[0];
}

/* 最小包含円 */
double SmallestEnclosingCircle(const vector<Point> &V) {
  int N = V.size();
  if(N <= 1) return 0;

  // 最小包含円の中心の候補
  vector<Point> CenterCandidate;
  for(int i = 0; i < N; i++) {
    for(int j = i + 1; j < N; j++) {
      // 最小包含円の円弧上に点が2つしかないの時
      CenterCandidate.push_back( (V[i] + V[j]) / 2 );
      for(int k = j + 1; k < N; k++) {
        if(!simple_ccw(V[i], V[j], V[k])) {
          // 三点が一直線上にある
          continue;
        }
        // 三点の外心が円の中心
        Point r = gaisin(V[i], V[j], V[k]);
        CenterCandidate.push_back(r);
      }
    }
  }

  double res = INF;
  rep(c, CenterCandidate.size()) {
    double tmp = 0.0;
    rep(v, V.size()) {
      // 中心からの距離が最大の点との距離が,包含円の半径になる
      chmax(tmp, norm2(V[v] - CenterCandidate[c]));
    }
    // 候補の包含円の中で,半径が最小の包含円が最小包含円になる.
    chmin(res, tmp);
  }
  return res;
}

// 直線a-bと点pの距離
long double distance(const Point& p, const Point& a, const Point& b) {
  long double A = b.y - a.y;
  long double B = a.x - b.x;
  long double C = a.y * b.x - b.y * a.x;
  return abs(A * p.x + B * p.y + C) / sqrt(A * A + B * B);
}

\end{lstlisting}

\color{white}
\section{knapsack}
\color{black}
\begin{lstlisting}[caption=knapsack]

int knapsack(int n, int W, vi w, vi v) {
  vvi dp(n + 1, vi (W + 1, 0));
  for(int i = 1; i <= n; i++) {
    for(int j = 1; j <= W; j++) {
      if(j - w[i] >= 0) {
        chmax(dp[i][j], dp[i - 1][j - w[i]] + v[i]);
      }
      chmax(dp[i][j], dp[i - 1][j]);
    }
  }
  return dp[n][W];
}

\end{lstlisting}

\color{white}
\section{kruskal}
\color{black}
\begin{lstlisting}[caption=kruskal]

// Union-Findも呼んで!!
struct kr_edge {
  int u; // 辺の片方,fromではないので二回辺を張る必要はない
  int v; // 辺のもう片方
  int cost;

  // コストの大小で順序定義
  bool operator<(const kr_edge& e) const {
    return cost < e.cost;
  }
};
class Kruskal{
  public:

  bool comp(const kr_edge& e1, const kr_edge& e2) { // sort関数の比較関数
    return e1.cost < e2.cost;
  }

  vector<kr_edge> es; // 辺の集合
  vector<kr_edge> minst; // 最小全域木に用いられる辺の集合
  int V, E; // 頂点数と辺数

  Kruskal(int v) {
    V = v;
  }

  void add(int v, int u, int cost){
    kr_edge e = {v, u, cost};
    es.push_back(e);
  }

  int kruskal(void) {
    sort(es.begin(), es.end()); // kr_edge.costが小さい順にソートされる
    UnionFind uni(V); //union-findの初期化
    int res = 0;
    for(int i = 0; i < es.size(); i++) {
      kr_edge e = es[i];
      if(uni.root(e.u) != uni.root(e.v)) {
        uni.connect(e.u, e.v);
        res += e.cost;
        minst.push_back(e);
      }
    }
    return res;
  }

  void show(void) {
    vvi v(V, vi(V, -1));
    for(int i = 0; i < minst.size(); i++) {
      v[minst[i].u][minst[i].v] = minst[i].cost;
      v[minst[i].v][minst[i].u] = minst[i].cost;
    }
    for(int i = 0; i < V; i++) {
      for(int j = 0; j < V; j++) {
        if(v[i][j] == -1) {
          printf(\"  __ \");
        } else {
          printf(\"%4d \" v[i][j]);
        }
      }
      cout << endl;
    }
  }
};

\end{lstlisting}

\color{white}
\section{lambdaSort}
\color{black}
\begin{lstlisting}[caption=lambdaSort]

template<class T>
void lambdaSort(vector<T> &v) {
  sort(all(v), [](auto const& l, auto const& r) {
    return l.fi > r.fi; // このbool式が成り立つ時入れ替える
  });
}

\end{lstlisting}

\color{white}
\section{lca}
\color{black}
\begin{lstlisting}[caption=lca]

class LCA {
  int n;
  int MAX;
  vvi doubling, v;

  void init() {
    rep(i, n) {
      for(auto j : v[i]) {
        doubling[0][j] = i;
      }
    }
    for(int i = 1; i < MAX; i++) {
      rep(j, n) {
        doubling[i][j] = doubling[i - 1][doubling[i - 1][j]];
      }
    }
    depth[0] = 0;
    dfs(0, -1);
  }

  void dfs(const int crrPos, const int befPos) {
    for(auto i : v[crrPos]) {
      if(i == befPos || depth[i] != -1) {
        continue;
      }
      depth[i] = depth[crrPos] + 1;
      dfs(i, crrPos);
    }
  }

public:
  vi depth;

  // vは0が根のbfs木にする.親->子のように辺を張る.
  LCA(vvi &_v) : v(_v), n(_v.size()) {
    MAX = ceil(log2(n));
    doubling = vvi(MAX, vi(n, 0));
    depth = vi(n, -1);
    init();
  }

  void show(const int height = 0) {
    rep(i, ((!height)?MAX:height)) {
      dump(doubling[i]);
    }
    dump(depth);
  }

  // ダブリングでVのnum個親の祖先を調べる
  int doublingNode(int V, const int num) {
    rep(i, MAX) {
      if((1LL<<i) & num) {
        V = doubling[i][V];
      }
    }
    return V;
  }

  int lca(int A, int B) {
    // Aのが深い位置にあるようにする
    if(depth[A] < depth[B]) {
      swap(A, B);
    }
    A = doublingNode(A, depth[A] - depth[B]);
    if(A == B) {
      return A;
    }

    for (int k = MAX - 1; k >= 0; k--) {
      if (doubling[k][A] != doubling[k][B]) {
        A = doubling[k][A];
        B = doubling[k][B];
      }
    }
    return doubling[0][A];
  }
};

\end{lstlisting}

\color{white}
\section{lcm}
\color{black}
\begin{lstlisting}[caption=lcm]

// gcdも呼ぶ!!!
ll lcm(ll a, ll b) { return a / gcd(a,   b) * b;}

\end{lstlisting}

\color{white}
\section{lcs}
\color{black}
\begin{lstlisting}[caption=lcs]

string lcs(string s, string t) {
  vvi dp(s.size() + 1, vi(t.size() + 1));
  
  for(int i = 0; i < s.size(); i++) {//LCS
    for(int j = 0; j < t.size(); j++) {
      if(s[i] == t[j]) {
        dp[i + 1][j + 1] = dp[i][j] + 1;
      }
      else{
        dp[i + 1][j + 1] = max(dp[i][j + 1], dp[i + 1][j]);
      }
    }
  }
  // 復元
  string ans = \"\";
  int i = (int)s.size(), j = (int)t.size();
  while (i > 0 && j > 0){
    // (i-1, j) -> (i, j) と更新されていた場合
    if (dp[i][j] == dp[i-1][j]) {
      --i; // DP の遷移を遡る
    }
    // (i, j-1) -> (i, j) と更新されていた場合
    else if (dp[i][j] == dp[i][j-1]) {
      --j; // DP の遷移を遡る
    }
    // (i-1, j-1) -> (i, j) と更新されていた場合
    else {
      ans = s[i-1] + ans;
      --i, --j; // DP の遷移を遡る
    }
  }
  return ans;
}

\end{lstlisting}

\color{white}
\section{lis}
\color{black}
\begin{lstlisting}[caption=lis]

int lis(vi& v) {
  vi dp(1, v[0]);
  for(int i = 1; i < v.size(); i++) {
    if(v[i] > dp[dp.size() - 1]) {
      dp.push_back(v[i]);
    } else {
      int pos = distance(dp.begin(), lower_bound(dp.begin(), dp.end(), v[i]));
      dp[pos] = v[i];
    }
  }
  return dp.size();
}

\end{lstlisting}

\color{white}
\section{mintcombination}
\color{black}
\begin{lstlisting}[caption=mintcombination]

#define MAX_MINT_NCK 201010
mint f[MAX_MINT_NCK], rf[MAX_MINT_NCK];

bool isinit = false;

void init() {
  f[0] = 1;
  rf[0] = 1;
  for(int i = 1; i < MAX_MINT_NCK; i++) {
    f[i] = f[i - 1] * i;
    // rf[i] = modinv(f[i].x);
    rf[i] = f[i].pow(f[i], MOD - 2);
  }
}

mint nCk(mint n, mint k) {
  if(n < k) return 0;
  if(!isinit) {
    init();
    isinit = 1;
  }
  mint nl = f[n.x]; // n!
  mint nkl = rf[n.x - k.x]; // (n-k)!
  mint kl = rf[k.x]; // k!
  mint nkk = (nkl.x * kl.x);

  return nl * nkk;
}

\end{lstlisting}

\color{white}
\section{mnmod}
\color{black}
\begin{lstlisting}[caption=mnmod]

// xのn乗%modを計算
ll mod_pow(ll x, ll n, ll mod = MOD) {
  ll res = 1;
  while(n > 0) {
    if(n & 1) res = res*x%mod;
    x = x*x%mod;
    n >>= 1;
  }
  return res;
}

\end{lstlisting}

\color{white}
\section{modint}
\color{black}
\begin{lstlisting}[caption=modint]

struct mint {
  ll x;
  mint(ll _x = 0) : x((_x % MOD + MOD) % MOD) {}

  /* 初期化子 */
  mint operator+() const { return mint(x); }
  mint operator-() const { return mint(-x); }

  /* 代入演算子 */
  mint& operator+=(const mint a) {
    if ((x += a.x) >= MOD) x -= MOD;
    return *this;
  }
  mint& operator-=(const mint a) {
    if ((x += MOD - a.x) >= MOD) x -= MOD;
    return *this;
  }
  mint& operator*=(const mint a) {
    (x *= a.x) %= MOD;
    return *this;
  }
  mint& operator/=(const mint a) {
    x *= modinv(a).x;
    x %= MOD;
    return (*this);
  }
  mint& operator%=(const mint a) {
    x %= a.x;
    return (*this);
  }
  mint& operator++() {
    ++x;
    if (x >= MOD) x -= MOD;
    return *this;
  }
  mint& operator--() {
    if (!x) x += MOD;
    --x;
    return *this;
  }
  mint& operator&=(const mint a) {
    x &= a.x;
    return (*this);
  }
  mint& operator|=(const mint a) {
    x |= a.x;
    return (*this);
  }
  mint& operator^=(const mint a) {
    x ^= a.x;
    return (*this);
  }
  mint& operator<<=(const mint a) {
    x *= pow(2, a).x;
    return (*this);
  }
  mint& operator>>=(const mint a) {
    x /= pow(2, a).x;
    return (*this);
  }

  /* 算術演算子 */
  mint operator+(const mint a) const {
    mint res(*this);
    return res += a;
  }
  mint operator-(const mint a) const {
    mint res(*this);
    return res -= a;
  }
  mint operator*(const mint a) const {
    mint res(*this);
    return res *= a;
  }
  mint operator/(const mint a) const {
    mint res(*this);
    return res /= a;
  }
  mint operator%(const mint a) const {
    mint res(*this);
    return res %= a;
  }
  mint operator&(const mint a) const {
    mint res(*this);
    return res &= a;
  }
  mint operator|(const mint a) const {
    mint res(*this);
    return res |= a;
  }
  mint operator^(const mint a) const {
    mint res(*this);
    return res ^= a;
  }
  mint operator<<(const mint a) const {
    mint res(*this);
    return res <<= a;
  }
  mint operator>>(const mint a) const {
    mint res(*this);
    return res >>= a;
  }

  /* 条件演算子 */
  bool operator==(const mint a) const noexcept { return x == a.x; }
  bool operator!=(const mint a) const noexcept { return x != a.x; }
  bool operator<(const mint a) const noexcept { return x < a.x; }
  bool operator>(const mint a) const noexcept { return x > a.x; }
  bool operator<=(const mint a) const noexcept { return x <= a.x; }
  bool operator>=(const mint a) const noexcept { return x >= a.x; }

  /* ストリーム挿入演算子 */
  friend istream& operator>>(istream& is, mint& m) {
    is >> m.x;
    m.x = (m.x % MOD + MOD) % MOD;
    return is;
  }
  friend ostream& operator<<(ostream& os, const mint& m) {
    os << m.x;
    return os;
  }

  /* その他の関数 */
  mint modinv(mint a) { return pow(a, MOD - 2); }
  mint pow(mint x, mint n) {
    mint res = 1;
    while (n.x > 0) {
      if ((n % 2).x) res *= x;
      x *= x;
      n.x /= 2;
    }
    return res;
  }
};

\end{lstlisting}

\color{white}
\section{ngcd}
\color{black}
\begin{lstlisting}[caption=ngcd]

// gcdも呼ぶ!!!
ll ngcd(vector<ll> a){
  ll d;
  d = a[0];
  for(int i = 1; i < a.size() && d != 1; i++) d = gcd(a[i], d);
  return d;
}

\end{lstlisting}

\color{white}
\section{nijihouteishiki}
\color{black}
\begin{lstlisting}[caption=nijihouteishiki]

/*
  aX^2+bX+c=0の解を求める
  出力はこんな感じ
  if(x1 == DBL_MIN)cout<<\"解なし\"<<endl;
  else if(x1==DBL_MAX)cout<<\"不定\"<<endl;
  else if(!i)cout<<x1<<\" , \"<<x2<<endl;
  else cout<<x1<<\" +- \"<<x2<<\"i\"<<endl;
*/
double x1, x2;
bool i = false;
void quadeq(double a, double b, double c){
  double d, x;
  if(a != 0){
    b /= a; c /= a;
    if(c != 0){
      b /= 2;
      d = b*b - c;
      if(d > 0){
        if(b > 0) x = -b - sqrt(d);
        else x = -b + sqrt(d);
        x1 = x; x2 = c/x;
      }else if(d < 0){
        x1 = -b; x2 = sqrt(-d); i = true;
      }else{
        x1 = x2 = -b;
      }
    }else{
      x1 = -b; x2 = 0;
    }
  }else if(b != 0){
    x1 = x2 = -c/b;
  }
  else if(c != 0) x1 = x2 = DBL_MIN;
  else x1 = x2 = DBL_MAX;
}

\end{lstlisting}

\color{white}
\section{nlcm}
\color{black}
\begin{lstlisting}[caption=nlcm]

// gcdも呼ぶ!!!
// lcmも呼ぶ!!!
ll nlcm(vector<ll> numbers) {
  ll l = numbers[0];
  for (int i = 1; i < numbers.size(); i++) {
    l = lcm(l, numbers[i]);
  }
  return l;
}

\end{lstlisting}

\color{white}
\section{primedissasembly}
\color{black}
\begin{lstlisting}[caption=primedissasembly]

map<ll, ll> prime;
void factorize(ll n) {
  for(ll i = 2; i * i <= n; i++) {
    while(n % i == 0) {
      prime[i]++;
      n /= i;
    }
  }
  if(n != 1) {
    prime[n] = 1;
  }
}

\end{lstlisting}

\color{white}
\section{printvector}
\color{black}
\begin{lstlisting}[caption=printvector]

template<class T>
void print_vector(vector<T> &v) {
  rep(i, v.size()) {
    if(!i) cout << v[i];
    else cout << \" \" << v[i];
  }
  cout << endl;
}

\end{lstlisting}

\color{white}
\section{sanjihouteishiki}
\color{black}
\begin{lstlisting}[caption=sanjihouteishiki]

// 三次方程式 ax^3+bx^2+cx+d=0を解く 
double ans1=0, ans2=0, ans3=0;
void cardano(double a, double b, double c, double d){
  double p, q, t, a3, b3, x1, x2, x3;
  b /= (3*a); c /= a; d /= a;
  p = b*b - c/3;
  q = (b*(c - 2*b*b) - d)/2;
  a = q*q - p*p*p;
  if(a == 0){
    q = cbrt(q); x1 = 2*q - b; x2 = -q - b;
    cout << \"x=\" << x1 << \" \" << x2 << \"(重解)\" << endl;
    ans1 = x1; ans2 = x2;
  }else if(a > 0){
    if(q > 0) a3 = cbrt(q + sqrt(a));
    else   a3 = cbrt(q - sqrt(a));
    b3 = p/a3;
    x1 = a3 + b3 - b; x2 = -0.5 + (a3 + b3) - b;
    x3 = fabs(a3 - b3)*sqrt(3.0)/2;
    cout << \"x=\" << x1 << \"; \" << x2 << \"+- \" << x3 << \"i\" << endl;
    ans1 = x1; ans2 = x2; ans3 = x3;
  }else{
    a = sqrt(p); t = acos(q/(p*a)); a *= 2;
    x1 = a*cos(t/3) - b;
    x2 = a*cos((t+2*M_PI)/3) - b;
    x3 = a*cos((t+4*M_PI)/3) - b;
    cout << \"x=\" << x1 << \" \" << x2 << \" \" << x3 << endl;
    ans1 = x1; ans2 = x2; ans3 = x3;
  }
}

\end{lstlisting}

\color{white}
\section{stringcount}
\color{black}
\begin{lstlisting}[caption=stringcount]

int stringcount(string s, char c) {
  return count(s.cbegin(), s.cend(), c);
}

\end{lstlisting}

\color{white}
\section{templete}
\color{black}
\begin{lstlisting}[caption=templete]

#include <algorithm>
#include <bitset>
#include <cassert>
#include <cctype>
#include <cfloat>
#include <climits>
#include <cmath>
#include <cstdio>
#include <deque>
#include <iomanip>
#include <iostream>
#include <list>
#include <map>
#include <queue>
#include <random>
#include <set>
#include <stack>
#include <string>
#include <unordered_set>
#include <vector>
#pragma GCC target(\"avx2\")
#pragma GCC optimize(\"O3\")
#pragma GCC optimize(\"unroll-loops\")
using namespace std;
typedef long double ld;
typedef long long int ll;
typedef unsigned long long int ull;
typedef vector<int> vi;
typedef vector<char> vc;
typedef vector<bool> vb;
typedef vector<double> vd;
typedef vector<string> vs;
typedef vector<ll> vll;
typedef vector<pair<int, int>> vpii;
typedef vector<pair<ll, ll>> vpll;
typedef vector<vi> vvi;
typedef vector<vvi> vvvi;
typedef vector<vc> vvc;
typedef vector<vs> vvs;
typedef vector<vll> vvll;
typedef pair<int, int> P;
typedef map<int, int> mii;
typedef set<int> si;
#define rep(i, n) for (ll i = 0; i < (n); ++i)
#define rrep(i, n) for (int i = 1; i <= (n); ++i)
#define irep(it, stl) for (auto it = stl.begin(); it != stl.end(); it++)
#define drep(i, n) for (int i = (n)-1; i >= 0; --i)
#define fin(ans) cout << (ans) << '\n'
#define STLL(s) strtoll(s.c_str(), NULL, 10)
#define mp(p, q) make_pair(p, q)
#define pb(n) push_back(n)
#define all(a) a.begin(), a.end()
#define Sort(a) sort(a.begin(), a.end())
#define Rort(a) sort(a.rbegin(), a.rend())
#define fi first
#define se second
// #include <atcoder/all>
// using namespace atcoder;
constexpr int dx[8] = {1, 0, -1, 0, 1, -1, -1, 1};
constexpr int dy[8] = {0, 1, 0, -1, 1, 1, -1, -1};
template <class T, class U>
inline bool chmax(T& a, U b) {
  if (a < b) {
    a = b;
    return 1;
  }
  return 0;
}
template <class T, class U>
inline bool chmin(T& a, U b) {
  if (a > b) {
    a = b;
    return 1;
  }
  return 0;
}
template <class T, class U>
ostream& operator<<(ostream& os, pair<T, U>& p) {
  cout << \"(\" << p.first << \" \" << p.second << \")\";
  return os;
}
template <class T>
inline void dump(T& v) {
  // if (v.size() > 100) {
  // cout << \"ARRAY IS TOO LARGE!!!\" << endl;
  // } else {
  irep(i, v) { cout << (*i) << ((i == --v.end()) ? '\n' : ' '); }
  // }
}
template <class T, class U>
inline void dump(map<T, U>& v) {
  if (v.size() > 100) {
    cout << \"ARRAY IS TOO LARGE!!!\" << endl;
  } else {
    irep(i, v) { cout << i->first << \" \" << i->second << '\n'; }
  }
}
template <class T, class U>
inline void dump(pair<T, U>& p) {
  cout << p.first << \" \" << p.second << '\n';
}
inline void yn(const bool b) { b ? fin(\"yes\") : fin(\"no\"); }
inline void Yn(const bool b) { b ? fin(\"Yes\") : fin(\"No\"); }
inline void YN(const bool b) { b ? fin(\"YES\") : fin(\"NO\"); }
const int INF = INT_MAX;
constexpr ll LLINF = 1LL << 61;
// constexpr ll MOD = 1000000007;  // 998244353; //
constexpr ld EPS = 1e-11;
void Case(int i) { printf(\"Case #%d: \" i); }
/* --------------------   ここまでテンプレ   -------------------- */

int main() {
  
}

\end{lstlisting}

\color{white}
\section{topologicalsort}
\color{black}
\begin{lstlisting}[caption=topologicalsort]

vvi G(1000); // グラフ(リスト)

// トポロジカルソート
void rec(int v, vector<bool> &seen, vector<int> &order) {
  seen[v] = true;
  for (int i= 0; i < G[v].size(); i++) {
    int next = G[v][i];
    if (seen[next]) continue; // 既に訪問済みなら探索しない
    rec(next, seen, order);
  }
  order.push_back(v);
}

vector<int> topo(int N) { // Nはノード数
  // 探索
  vector<bool> seen(N, 0); // 初期状態では全ノードが未訪問
  vector<int> order; // トポロジカルソート順
  for (int v = 0; v < N; ++v) {
    if (seen[v]) continue; // 既に訪問済みなら探索しない
    rec(v, seen, order);
  }
  reverse(order.begin(), order.end());
  return order;
}

\end{lstlisting}

\color{white}
\section{unionfind}
\color{black}
\begin{lstlisting}[caption=unionfind]

class UnionFind {
public:
  // 親の番号を格納.親だった場合は-(その集合のサイズ)
  vector<int> Parent;
  // 重さの差を格納
  vector<ll> diffWeight;

  UnionFind(const int N) {
    Parent = vector<int>(N, -1);
    diffWeight = vector<ll>(N, 0);
  }

  // Aがどのグループに属しているか調べる
  int root(const int A) {
    if (Parent[A] < 0) return A;
    int Root = root(Parent[A]);
    diffWeight[A] += diffWeight[Parent[A]];
    return Parent[A] = Root;
  }

  // 自分のいるグループの頂点数を調べる
  int size(const int A) {
    return -Parent[root(A)];
  }

  // 自分の重さを調べる
  ll weight(const int A) {
    root(A); // 経路圧縮
    return diffWeight[A];
  }

  // 重さの差を計算する
  ll diff(const int A, const int B) {
    return weight(B) - weight(A);
  }

  // AとBをくっ付ける
  bool connect(int A, int B, ll W = 0) {
    // Wをrootとの重み差分に変更
    W += weight(A);
    W -= weight(B);

    // AとBを直接つなぐのではなく、root(A)にroot(B)をくっつける
    A = root(A);
    B = root(B);

    if (A == B) {
      //すでにくっついてるからくっ付けない
      return false;
    }

    // 大きい方(A)に小さいほう(B)をくっ付ける
    // 大小が逆だったらひっくり返す
    if (size(A) < size(B)) {
      swap(A, B);
      W = -W;
    }

    // Aのサイズを更新する
    Parent[A] += Parent[B];
    // Bの親をAに変更する
    Parent[B] = A;

    // AはBの親であることが確定しているのでBにWの重みを充てる
    diffWeight[B] = W;

    return true;
  }
};

\end{lstlisting}

\color{white}
\section{warshallfloyd}
\color{black}
\begin{lstlisting}[caption=warshallfloyd]

template <typename T>
class WAR_FLY {
public:
  vector<vector<T>> d; // 辺の数
  int V; // 頂点の数
  
  WAR_FLY(int n) {
    V = n;
    d = vector<vector<T> > (n,vector<T>(n));
    for(int i = 0; i < V; i++) {
      for(int j = 0; j < V; j++) {
        if(i == j) {
          d[i][j] = 0;
        }
        else {
          d[i][j] = LLINF;
        }
      }
    }
  }

  void warshall_floyd(void) {
    for(int k = 0; k < V; k++) {
      for(int i = 0; i < V; i++) {
        for(int j = 0; j < V; j++) {
          d[i][j] = min((ll)d[i][j], (ll)d[i][k] + (ll)d[k][j]);
        }
      }
    }
  }

  void add(int from, int to, T cost) {
    d[from][to] = cost;
  }

  bool is_negative_loop(void) {
    for(int i = 0; i < V; i++) {
      if(d[i][i] < 0) return true;
    }
    return false;
  }

  void show(void) {
    for(int i = 0; i < V; i++) {
      for(int j = 0; j < V; j++) {
        cout << d[i][j] << \" \";
      }
      cout << endl;
    }
  }
};

\end{lstlisting}

\color{white}
\section{yakusuenum}
\color{black}
\begin{lstlisting}[caption=yakusuenum]

vector<ll> enum_div(ll n) {
  vector<ll> ret;
  for(ll i = 1 ; i*i <= n ; ++i){
    if(n%i == 0) {
      ret.push_back(i);
      if(i != 1 && i*i != n){
        ret.push_back(n/i);
      }
    }
  }
  return ret;
}

\end{lstlisting}

\color{white}
\section{zip}
\color{black}
\begin{lstlisting}[caption=zip]

map<ll, ll> zip;
int compress(vector<ll> x) {
  int unzip[x.size()];
  sort(x.begin(), x.end());
  x.erase(unique(x.begin(),x.end()),x.end());
  for(int i = 0; i < x.size(); i++){
    zip[x[i]] = i;
    unzip[i] = x[i];
  }
  return x.size();
}

\end{lstlisting}

\end{document}
